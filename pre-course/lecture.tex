\documentclass[a4paper, markcase = upper, parskip = true, fleqn]{scrartcl}

%%%%%%%%%%%%%%
%%% layout %%%
%%%%%%%%%%%%%%

\newlength{\hmargin}
\newlength{\lmargin}
\newlength{\rmargin}
\newlength{\fmargin}
\setlength{\hmargin}		{2cm}
\setlength{\lmargin}		{2cm}
\setlength{\rmargin}		{2cm}
\setlength{\fmargin}		{2cm}

\usepackage{layout}
\usepackage[top=\hmargin, left=\lmargin, right=\rmargin, bottom=\fmargin]{geometry}
\usepackage{scrlayer-scrpage}
\pagestyle{scrheadings}
\automark[section]{section}
\ihead{\headmark}
\ohead*{\pagemark}
\chead{}
\cfoot{}

\setlength\parindent{1em}
\newcommand{\saveparinfos}{%
	\edef\myindent{\the\parindent}%
	\edef\myparskip{\the\parskip}%
}
\saveparinfos
\newcommand{\pardefs}{%
	\setlength{\parindent}{\myindent}%
	\setlength{\parskip}{\myparskip}%
}


%%%%%%%%%%%%%%%%%%%%%%%%%%
%%% necessary packages %%%
%%%%%%%%%%%%%%%%%%%%%%%%%%

%math and physics
\usepackage{amsmath}
\usepackage{amsfonts}
\usepackage{amssymb}
\usepackage{mathtools}
\usepackage{siunitx}
\usepackage{tikz}
\newcommand{\qed}{\tag{$\square$}}
%not math
\usepackage[T1]{fontenc}
\usepackage{enumitem}


%%%%%%%%%%%%%%%%%%%%%%%%%%%%%%
%%% not necessary packages %%%
%%%%%%%%%%%%%%%%%%%%%%%%%%%%%%

\usepackage[ngerman]{babel}
\usepackage{ulem}


%%%%%%%%%%%%%%
%%% colors %%%
%%%%%%%%%%%%%%

\usepackage[many]{tcolorbox}	%for COLORED BOXES
\definecolor{gadse-black}	{RGB}{000, 000, 000}
\definecolor{gadse-white}	{RGB}{255, 255, 255}
\definecolor{gadse-background}	{RGB}{208, 197, 112}
\definecolor{gadse-light-green}	{RGB}{113, 212, 083}
\definecolor{gadse-dark-green}	{RGB}{048, 109, 000}
\definecolor{gadse-red}		{RGB}{255, 016, 000}
\definecolor{gadse-pink}	{RGB}{255, 031, 206}
\definecolor{gadse-orange}	{RGB}{255, 139, 000}
\definecolor{gadse-yellow}	{RGB}{255, 197, 017}
\definecolor{gadse-light-blue}	{RGB}{079, 160, 183}
\definecolor{gadse-another-blue}{RGB}{051, 110, 217}
\definecolor{gadse-dark-blue}	{RGB}{051, 030, 181}


%%%%%%%%%%%%%%%%%%%%%%%%%%%%%%%%
%%%    commands to create    %%%
%%%  the commands for these  %%%
%%% environements I am using %%%
%%%%%%%%%%%%%%%%%%%%%%%%%%%%%%%%

\newcommand{\myboxenvironment}[4]{ % tcolorbox-environment %
	\newtcolorbox{cache#1#2box}[1]{
		title		= ##1,
		fonttitle	= \bfseries,
		colback		= #3,
		colframe	= #4,
		sharp corners,
		rounded corners	= northeast,
		rounded corners	= southeast,
		arc		= 2pt,
		outer arc	= 2pt,
		boxrule = 0pt,
		leftrule = 4pt,
		breakable
	}
}
\newcommand{\cachemyenvironment}[4]{ % command to define \#1#2number, #1#2 tcolorbox, #1#2box %
	%\setlength\parindent{1em}
	\newcounter{#1#2number}[#1section]
	\expandafter\renewcommand\csname the#1#2number\endcsname{\csname the#1section\endcsname.\arabic{#1#2number}}
	\myboxenvironment{#1}{#2}{#3}{#4}
	\newenvironment{#1#2box}[1][~]{
		\begin{cache#1#2box}{\MakeUppercase#2 \csname the#1#2number\endcsname\quad##1}%
			\refstepcounter{#1#2number}
			\setlength\parindent{1em}
		}{
		\end{cache#1#2box}
	}
}


%%%%%%%%%%%%%%%%%%%%%%%%%%%%%%%%
%%%    command to create     %%%
%%% these environments I use %%%
%%%%%%%%%%%%%%%%%%%%%%%%%%%%%%%%

\newcommand{\myenvironment}[3]{
	\cachemyenvironment{}{#1}{#2}{#3}%
	\cachemyenvironment{sub}{#1}{#2}{#3}
}

\myenvironment{example}		{gadse-orange}%
				{gadse-red}
\myenvironment{exercise}	{gadse-orange!50!gadse-yellow}%
				{gadse-red!50!gadse-orange}
\myenvironment{problem}		{gadse-yellow}%
				{gadse-orange}
\myenvironment{definition}	{gadse-pink!50!gadse-white}%
				{gadse-pink}
\myenvironment{theorem}		{gadse-dark-green!50!gadse-light-green}%
				{gadse-dark-green!100!gadse-ligth-green}
\myenvironment{lemma}		{gadse-dark-green!25!gadse-light-green}%
				{gadse-dark-green!75!gadse-light-green}
\myenvironment{conjecture}	{gadse-light-green}%
				{gadse-dark-green!25!gadse-light-green}
\myenvironment{proof}		{gadse-light-green!50!gadse-white}%
				{gadse-light-green!100!gadse-white}


\newenvironment{indentpar}{
	\hfill\begin{minipage}{\dimexpr\textwidth-2em}%
		\pardefs%
	}{
	\end{minipage}
}

%%%%%%%%%%%%%%%%%%%%%%%%%%%%
%%% sets #2numbers to #1 %%%
%%%%%%%%%%%%%%%%%%%%%%%%%%%%

\newcommand{\setnumbers}[2]{
	\setcounter{#2examplenumber}	{#1}
	\setcounter{#2exercisenumber}	{#1}
	\setcounter{#2problemnumber}	{#1}
	\setcounter{#2definitionnumber}	{#1}
	\setcounter{#2theoremnumber}	{#1}
	\setcounter{#2lemmanumber}	{#1}
	\setcounter{#2conjecturenumber}	{#1}
	\setcounter{#2proofnumber}	{#1}
}


%%%%%%%%%%%%
%%% tikz %%%
%%%%%%%%%%%%

\newcommand{\circled}[1]{
	\tikz[baseline=(char.base)]{
		\node[shape=circle, draw, inner sep = 2pt] (char) {#1};
	}
}
%%%%%%%%%%%%%
%%% title %%%
%%%%%%%%%%%%%

%\usepackage{titlesec}
%\titleformat{\section}{\bfseries}{\setnumbers{1}{}\thesection . }{0pt}{}{}
%\titleformat{\subsection}{\bfseries}{\setnumbers{1}{sub}\thesubsection . }{0pt}{}{}

\renewcommand*{\sectionformat}{\setnumbers{1}{}\thesection.\enskip}
\renewcommand*{\subsectionformat}{\setnumbers{1}{sub}\thesubsection.\enskip}


%%%%%%%%%%%%%
%%% specs %%%
%%%%%%%%%%%%%

\title{pre-course lecture}
\author{Elias Gestrich}
\date{\today}

\begin{document}

\setlength{\hoffset}		{0.5\lmargin-1in}
\setlength{\voffset}		{0.5\hmargin-1in}
\setlength{\oddsidemargin}	{0.5\lmargin}
\setlength{\topmargin}		{-0.5cm}
\setlength{\headheight}		{1cm}
\setlength{\headsep}		{0.5\hmargin-0.5cm}
%
\setlength{\textheight}		{\paperheight-\hmargin-\fmargin}
\setlength{\textwidth}		{\paperwidth-\lmargin-\rmargin}
\setlength{\linewidth}		{\textwidth}
%\setlength{\headwidth}		{\textwidth}
%
\setlength{\marginparsep}	{0.25cm}
\setlength{\marginparwidth}	{1.5cm}

\maketitle
\thispagestyle{empty}


%%%%%%%%%%%%%%%%
%%% 1. Day 1 %%%
%%%%%%%%%%%%%%%%

\section{Day 1}
Argumente/Begründungen/Beweise (Herz der Mathematik)\\
Beweise sind ``immer wahr''\\
Beweise helfen beim verstehen\\
Beweise ``zähmen die Unendlichkeit''\par

\begin{problembox}
	Wie lange benötigt man zum Zersägen eines $\qty{7}{\meter}$-langen Baumstamms in $\qty{1}{\meter}$-Stücke, wenn jeder Schnitt eine halbe Minute dauert?\par
	Lsg.: $\qty{3}{\minute}$\par
	Zwischenziel: 6 Schritte $\rightarrow$ Mustererkennung $n-1$ Schnitte für $n$ Teiler/Meter\par
	\begin{proofbox}
		Jeder Schnitt erhöht die Anzahl der Stücke um eins.\par
		Anfangs: $1$ Stück, am Ende $7$ $\Rightarrow$ man muss \[7 - 1 = 6 \text{ Schnitte machen. usw.}\]
	\end{proofbox}
\end{problembox}

$\mathbb{N} = \{1, 2, 3, ...\} \text{ natürliche Zahlen}$\\
$\mathbb{N}_0 = \{0, 1, 2, ...\} \text{ natürliche Zahlen mit der Null}$\par

\begin{problembox}
	\begin{conjecturebox}
		Wenn zwei natürliche Zahlen gerade sind, dann ist auch ihre Summe gerade
	\end{conjecturebox}
	Seien $t_1, t_2 \in \mathbb{N}$
	\[ a = 2t_1 \]
	\[ b = 2t_2 \]
	also gilt:
	\[ a + b = 2t_1 + 2_t2 = 2(t_1+t_2) \implies \text{ gerade Zahl}\]
\end{problembox}

\begin{problembox}
	Mit wievielen Nullen endet $100!$?\par
	Primfaktorzerlegung:\\
	$20$ durch $5$ teilbare Zahlen\\
	$4$ durch $5^2$ teilbare Zahlen\par
	$k$ Nullen am Ende einer natürlichen Zahl in Dezimalschreibweise: Zahl ist durch $10^k = \left( 2 \times 5 \right)^k = 2^k  \times 5^k$ teilbar.
	\[5, 10, 15, ...\]
	also bei
	\[1 \times 5, 2 \times 5, 3 \times 5, ..., 20 \times 5 \Rightarrow \qty{20}{\text{Stück}}\]
	welche liefern $2$ Fünfen
	\[1 \times 5^2, 2 \times 5^2, 3 \times 5^2, 4 \times 5^2, \text{\hbox{\sout{$5 \times 5^2$}}}\]\par
	Anzahl der Nullen, mit denen $100!$ enden, ist die größte natürliche Zahl $k \in \mathbb{N}$, für die
	\[100! \text{ durch } 10^k \text{ teilbar ist.}\]
	Weil $10 = (2 \times 5)$, ist das gleich der größten ganzen Zahl $k$, für die $100!$ durch $2^k \times 5^k$ teilbar ist,
	\[\text{also durch } 5^k \text{ und } 2^k\]
	Die $5$ tritt als Faktor genau in den $20$ Zahlen
	\[5, 10, 15, ..., 100,\]
	und in den $4$ Zahlen
	\[25, 50, 75, 100\]
	doppelt vor.\\
	Somit folgt: die $5$ tritt $20 + 4 = 24$ mal als Faktor in $100!$ auf.\\
	Die $2$ tritt als Faktor in $50$ Zahlen auf:
	\[2, 4, 6, 8, ..., 100\]
	(und einigen mehrfach)\\
	$\Rightarrow$ Insgesamt endet $100!$ mit $24$ Nullen.
\end{problembox}

\begin{problembox}
	Es sei $n \in \mathbb{N}$. Berechne die Summe der ersten $n$ natürlichen Zahlen, also
	\begin{alignat*}{7}
		&~{}	&&1		&&+ 2		&&+ 3		&&+ ... &&+ n			&&\coloneqq S\\
		&+{} 	&&n		&&+ n - 1	&&+ n - 2	&&+ ... &&+ 1			&&= S\\
		\hline
		&={} 	&&(n+1)		&&+(n+1)	&&+(n+1)	&&+ ... &&+(n+1)		&&= S\\
		&={} 	&&~		&&~		&&~		&&~	&&n\times (n + 1)	&&= S
	\end{alignat*}
\end{problembox}

\section{Tag 2}
$\mathbb{P} = \text{Menge aller Primzahlen}$\\
$\mathbb{P} = {2, 3, 5, 7, ...}$\\
Offene Fragen: Gibt es eine Formel für Primzahlen?\\
Fermat (1637):\\
\begin{minipage}{0.5\textwidth}
	\setlength{\parindent}{1em}
	\begin{conjecturebox}
		$F(n) = 2^{2^n} + $ ist eine Primzahl für jedes $n \in \mathbb{N}$
		\begin{alignat*}{3}
			F(1) &= 2^{2^1} + 1 &&= 2^2 + 1 &&= 5\\
			F(2) &= 2^{2^2} + 1 &&= 2^4 + 1 &&= 17\\
			F(3) &= 2^{2^3} + 1 &&= 2^8 + 1 &&= 257\\
			F(4) &= 2^{2^4} + 1 &&= 2^16 + 1 &&= 65537
		\end{alignat*}
	\end{conjecturebox}
\end{minipage}
\begin{minipage}{0.05\textwidth}
	~
\end{minipage}
\begin{minipage}{0.4\textwidth}
	\pardefs
	100 Jahre später Euler:\\
	$651$ teilt $F(5)$,\\
	also ist
	\[F(5) = 4294967297\]
	\underline{keine} Primzahl.
\end{minipage}

\section{Vorkurs}
\begin{itemize}
	\item mathematische Probleme formulieren
	\item Lösungen finden (Argumentieren) Beweisen
\end{itemize}

\section{Mengen (Warm-up)}
Skatspiel\par
Situation: Aus gemischtem Skatspiel werden zwei Karten gezogen. Sind es zwei Karten gleicher Farbe oder zwei Karten mit gleichem Wert, dann gewinnen wir.\par
Symbole: Kreuz, Pik, Herz, Karo\par
Werte: 7, 8, 9, 10, B, D, K, Ass\par
Definitionssymbol: ``$\coloneqq$''\par
\begin{examplebox}
	\begin{align*}
		11 &\coloneqq \text{Bube}	& a &\coloneqq \text{Kreuz}\\
		12 &\coloneqq \text{Dame}	& b &\coloneqq \text{Pik}\\
		13 &\coloneqq \text{König}	& c &\coloneqq \text{Herz}\\
		14 &\coloneqq \text{Ass}	& d &\coloneqq \text{Karo}
	\end{align*}
	$\text{Kartenmenge} = \left\{a7, a8, ..., a14, b7, b8, ..., b14, c7, ..., d14\right\}$\\
	$\left\{\text{Kreuz, 7}\right\} = \text{Karte Kreuz 7}$
	\begin{alignat*}{4}
		\text{Kartenmenge} \coloneqq \{	&\{1, 7\}, &&\{2, 7\}, &&\{3, 7\}, &&\{4, 7\}\\
		~				&\{1, 8\}, &&\{2, 8\}, &&\{3, 8\}, &&\{4, 8\}, ...,\\
		~				&\{1, 14\}, &&\{2, 14\}, &&\{3, 14\}, &&\{4, 14\}\}
	\end{alignat*}
\end{examplebox}


%%%%%%%%%%%%%%%%%%%%%%%%%%%%%%%%%%%%%%%%%%%%%%%%
%%% 5. Mengen, Teilmengen, Mengenoperationen %%%
%%%%%%%%%%%%%%%%%%%%%%%%%%%%%%%%%%%%%%%%%%%%%%%%

\section{Mengen, Teilmengen, Mengenoperationen}

\begin{definitionbox}
	(naive Definition) Eine Ansammlung von (mathematischen) Objekten heißt \textbf{Menge}. Ein Mitglied dieser Ansammlung heißt \textbf{Element} der Menge.
\end{definitionbox}

Ist $a$ ein Element der Menge $A$, so schreibt man $a \in A$\par
Gehört $a$ nicht zur Menge $A$, so schreibt man $a \notin A$\\
Mit $\vert A \vert$ bezeichnet man die Anzahl der Elemente in $A$

\begin{examplebox}
	\[\vert\mathbb{N}\vert=\infty\]
\end{examplebox}

Zwei Arten, Mengen zu beschreiben:

\begin{itemize}
	\item aufzählende Mengenschreibweise $A = \{2, 4, 6, ...\}$, $B = \{2, 8, 9\}$
	\item beschreibende Mengenschreibweise $A = \{x \in \mathbb{N} \vert 3 \leq x \leq 8\}$, $[a, b] \coloneqq \{z \in \mathbb{R} \vert a \leq z \leq b \} a,b \in \mathbb{R}$
\end{itemize}

\begin{exercisebox}
	\begin{align*}
		M &= \text{Menge aller ganzzahlingen Vielfachen von 42}\\
		~ &= \{x \in \mathbb{Z} : 42\vert x\} = \{42\times x : x \in \mathbb{Z}\}
	\end{align*}
	\begin{definitionbox}
		Es sei $A$ eine Menge\\
		Eine Menge $B$ heißt Teilmenge von $A$, in Zeichen: $B \subseteq A$,\\
		wenn jedes Element von $B$ auch ein Element von $A$ ist
	\end{definitionbox}
	\begin{definitionbox}
		Es seien $A, B$ Teilmengen einer Menge $M$\\
		Schnittmenge von $A$ und $B$ ist $A \cap B \coloneqq \{x \in M : x\in A \wedge x \in B\}$\\
		Vereinigungsmenge von $A$ und $B$ ist $A \cup B \coloneqq \{x \in M : x\in A \vee x \in B\}$\\
		Differenz von $A$ und $B$ ist $A \backslash B \coloneqq \{x \in M : x\in A \wedge x \notin B\}$\\
		Komponent von $A$ in $M$ ist die Menge $A^c \coloneqq \{x \in M : x \notin A \} = M \backslash A\}$\\
		genau dann wenn $A \subseteq B$ und $B \subseteq A$, dann $A = B$
	\end{definitionbox}
\end{exercisebox}
Jede beliebeige Menge $A$ hat die folgenden Teilmengen $\emptyset, A$ d.h. es git $\emptyset \subseteq A,A \subseteq A$\\
$\{\_\} \notin \{\_,\_\}$\\
$\_ \in \{\square, \circ, \triangle\}$ (bzw. $\_ \in \{\_, \_, \_\}$)\\
$\{1, 2, 3\} = \{3, 2, 1\} = \{2, 1, 3\} = ...$
\begin{exercisebox}
	Gilt $\{1, 2, 1, 1\} = \{1, 2\}$?\\
	\indent\indent ja
\end{exercisebox}
\begin{exercisebox}
	Falls $\{a\} = \{b\}$, dann folgt $a = b$\\
	Falls $\{a\} = \{a, b\}$, dann folgt $a = b$
\end{exercisebox}
$\{1, 2\}$ hat Teilmengen $\emptyset, \{1\}, \{2\}, \{1, 2\}$\\
Eine Menge mit $n$ Elementen hat $2^n$ Teilmengen.
\begin{definitionbox}
	Potenzmenge einer Menge\\
	Sei $A$ eine Menge, dann heißt die Menge
	\[\mathcal{P}(A)\]
	aller Teilmengen von $A$ die \underline{Potenzmenge von $A$}
\end{definitionbox}
$\mathcal{P}(\emptyset) = \{\emptyset\}\:\mathcal{P}(\{1\}) = \{\emptyset, 1\}$
\begin{definitionbox}
	(kartesisches Produkt von zwei Mengen $A$ und $B$)\\
	Das kartesische Produkt von zwei Mengen $A$ und $B$ ist definiert durch
	\[A \times B \coloneqq \{(x,y): x \in A \wedge y \in B\}\]
	In $A \times B$ sind zwei Elemente $(x_1, y_1), (x_2, y_2)$ genau dann gleich, wenn $x_1 = x_2$ und $y_1 = y_2$
	\[\mathbb{R}^2 \coloneqq \mathbb{R} \times \mathbb{R} = \{(x, y): x, y \in \mathbb{R}\]
	\indent\indent $(1,2) \in \mathbb{R}^2 \qquad (1,2) \neq (2,1)$\\
	\indent\indent $(2,1) \in \mathbb{R}^2$
\end{definitionbox}

\section{Prof. Junk}
Auf einem Baum saß ein Rabe mit einem Käse im Schnabel, als ein Fuchs vorbeikommt.\\
Sei $B \in \text{Bäume}$ und $R \in \text{Raben}$ sitze auf $B$ mit $K \in \text{Käsestückchen}$ im Schnabel, als $F \in \text{Füchse}$ vorbeikommt.\par
Er überlegte wie er an den Käse kommt. Da sagte er zu dem Raben, ...\\
$F$ überlegte wie $F$ an $K$ kommt. Da sagte $F$ zu $R$, ...\par

allgemeine mathematische Situationen bestehen aus
\begin{itemize}
	\item Eine Liste von Namen für mathematische Objekte
	\item Eine Liste von geltenden Aussagen über diese Objekte
\end{itemize}
Was kann man damit tun?
\begin{itemize}
	\item Namen kann man austauschen ohne Bedeutungsänderung
	\item Situationen können \underline{eintreten}; konkrete Situationen können auf allgemeine Situationen passen
\end{itemize}

\begin{examplebox}
	$A \coloneqq \{ x \in \mathbb{N} : 3 \leq x \leq 8\}$\\
	Schreibweise beschreib eine Menge, indem sie zwei Zugehörigkeits\underline{regeln} kodiert
	\begin{itemize}
		\item besteht aus zwei allgemeinen Situationen:
			\begin{itemize}
				\item Vorraussetzung
				\item Folgerung
			\end{itemize}
	\end{itemize}
	Regel 1: Sei $x$ ein Objekt\\
	Es gelte $x \in \mathbb{N}; 3 \leq; 8 \geq x \rightarrow$ Dann gilt $x \in A$\\
	Regel 2: Sei $x$ ein Objekt\\
	Es gelte  $x \in A\rightarrow$ Dann gilt $x \in \mathbb{N}; 3 \leq; 8 \geq x$\\
	\indent Gilt: $9 \in A$:\quad$x \leftarrow 9$
	\indent\indent $9 \in \mathbb{N}, 9 \geq 3, 9 \leq 8$\\
	\indent\indent Vorraussetzung tritt nicht ein!
	\indent Gilt: $9 \notin A$
	\indent\indent $9 \in \mathbb{N}, 9 \geq 3, 9 \leq 8 \rightarrow$ Wiederspruch\\
	\indent\indent\indent $9 \nleq 8\rightarrow$ Gegenteil von Annahme stimmt
\end{examplebox}

\section{Aussagen (Logik)}
Unsere mathematische Ergebnisse formulieren wir als (mathematische) Aussagen.\\
Als Aussage bezeichnen wir einen Ausdruck, der entweder wahr oder falsch sein kann.\\
$\rightsquigarrow$ Wahrheitswert der Aussage: wahr, w, $1$; falsch, f, $0$
\begin{examplebox}
	\begin{align*}
		A &\coloneqq \text{ ``$1+2=3$''}\\
		B &\coloneqq \text{ ``$5$ ist eine negative Zahl''}\\
		C &\coloneqq \text{ ``Jede gerade Zahl $n > 2$ kann als Sume zweier Primzahlen geschrieben werden.''}
	\end{align*}
\end{examplebox}


%%%%%%%%%%%%%%%%%%%%%%
%%% 8. Aussageform %%%
%%%%%%%%%%%%%%%%%%%%%%

\section{Aussageform}

\begin{examplebox}
	Sei
	\[ A(n) \coloneqq n \text{ ist gerade}\]
	Dann ist
	\[ A(1) \text{ eine (falsche) Aussage}\]
	\[ A(2) \text{ eine (wahre) Aussage}\]
\end{examplebox}

Eine Aussageform ist eine Äußerung, die eine (oder mehrere) Variablen enthält und zu einer Aussage wird, wenn man zulässige Objekte für diese Variablen einsetzt.\\
Es seien $A$ und $B$ Aussagen.\par
Die Konjunktion (Und-Aussage) von $A$ und $B$ ist die Aussage
\[ A \land B \text{ ``$A$ und $B$''}\]
die genau dann wahr ist, wenn $A$ und $B$ gleichzeitig wahr sind.
\[ A \coloneqq \text{ ``$24$ ist gerade''}\]
\[ B \coloneqq \text{ ``$24$ ist durch $3$ teilbar''}\]
\[ C \coloneqq \text{ ``$24$ ist durch $5$ teilbar''}\]
\[ A \land B = 1\quad A \land C = 0\]
Die Disjunktion (Oder-Aussage) von $A$ und $B$ ist Aussage
\[A \lor B \text{ ``$A$ oder $B$''}\]
die genau dann wahr ist, wenn mind. eine der beiden Aussagen $A$ und $B$ wahr ist.\par
Die Negation von $A$ ist die Aussage
\[ \neg A \text{ ``nicht A''}\]
die genau dann wahr ist, wenn $A$ falsch ist.\par
Implikation (Folgerung) ``Wenn $A$, dann $B$'' ist die Aussage
\[ A \implies B \text{ ``aus $A$ folgt $B$'', $A$ impliziert $B$'')}\]
die genau dann falsch ist, wenn $A$ wahr ist und $B$ falsch ist und sonst ist sie immer wahr.\par
Äquivalenz ``A genau dann, wenn B'' ist die Aussage
\[ A \iff B \]
die genau dann wahr ist, wenn $A$ und $B$ beide wahr sind oder $A$ und $B$ beide falsch sind.\par
Eine zusammengesetzte Aussage heißt \underline{Tautologie,} falls sie immer wahr ist, unabhängig davon, welche Wahrheitswerte die verknüpften Einzelaussagen haben.\par

Ist  $A(n)$ eine Aussageform mit Wertebereich $M$, so können wir folgende Aussagen betrachten:
\begin{itemize}
	\item für alle $n \in M$ gilt: $A(n)$\\
		in Zeichen: $\forall n \in M : A(n)$ z.B. $\forall n \in \mathbb{N}: 2 \mid n$\\
		$\rightarrow$ Allaussage ($\forall \leftarrow$ Allquantor)
	\item Es gibt ein $n \in M$, für das $A(n)$ gilt\\
		in Zeichen $\exists n \in M : A(n)$\\
		\quad $\exists! n \in M : A(n)$ ``es gibt genau ein $n \in M, ...$''
	\item Eine Aussage der Form
		\[\forall n \in M : A(n)\]
		ist immer wahr, wenn $M = \{\}$
	\item Reihenfolge der Quantoren spielt eine Rolle
		\[A : \forall n \in \mathbb{N}: (\exists m \in \mathbb{N}: m > n)\text{ (wahr)}\]
		\[B : \exists m \in \mathbb{N}: \forall n \in \mathbb{N}: m > n\text{ (falsch)}\]
\end{itemize}


%%%%%%%%%%%%%%%%%%%%%%%%%%%%%%%%
%%% 9. Negation von Aussagen %%%
%%%%%%%%%%%%%%%%%%%%%%%%%%%%%%%%

\section{Negation von Aussagen}
$\neg (\exists n \in M : A(n)) \iff \forall n \in M : \neg A(n)\quad\text{ist eine Tautologie}$
d.h. die Negation der Aussage
\[ \exists n \in M : A(n) \]
ist logisch Äquivalent zu der Aussage
\[ \forall n \in M : \neg A(n) \]

$\neg ( \forall n \in M : A(n)) \iff \exists n \in M : \neg A(n)\quad\text{ist eine Tautologie}$
d.h. die Negation der Aussage
\[ \forall n \in M : A(n) \]
ist logisch Äquivalent zu der Aussage
\[ \exists n \in M : \neg A(n) \]

\section{---}
\begin{theorembox}
	\begin{enumerate}[label=\arabic*)]
		\item Äquivalenzprinzip\\
			Ist A wahr und ist $A \iff B$, dann ist auch $B$ wahr
		\item Ableitungsregel\\
			$(A \land (A \implies B)) \implies B$ ist eine Tautologie
		\item Syllogismusregel\\
			$((A \implies B) \land (B \implies C)) \implies A \;\implies\; C$ ist eine Tautologie
		\item Kontraposition\\
			$(A \implies B) \iff (\neg B \implies \neg A)$ ist eine Tautologie
		\item Ringschluss
			\begin{equation*}
				((A \iff B) \land (B \iff C) \land (A \iff C))
			\end{equation*}
			\begin{equation*}
				\iff
			\end{equation*}
			\begin{equation*}
				((A \iff B) \land (B \iff C) \land (A \iff C))
			\end{equation*}
	\end{enumerate}
\end{theorembox}


%%%%%%%%%%%%%%%%%%%%%%%%%%%%%%%%%%%%
%%% 11. Beweise und Beweisformen %%%
%%%%%%%%%%%%%%%%%%%%%%%%%%%%%%%%%%%%
\section{Beweise und Beweisformen}

Ein Beweis ist eine logisch vollständige Begründung einer Aussage. Oft möchten wir Aussagen vom Typ ``Wenn A, dann B'' zeigen.\\
Aussagen lassen sich meist in folgende Form bringen
\begin{itemize}
	\item Vorraussetzung z.B. Sei $a \in \mathbb{N}$.
	\item Behauptung\quad\quad\quad\quad dann ist $2 = \text{gerade}$
\end{itemize}

%%% Direkter Beweis %%%
\subsection{Direkter Beweis}
Statt $A \implies B$ zu zeigen, zeigen wir $A \implies A_1 \implies A_2 \implies ... \implies B$\par
Situation: $\mathbb{N} = \{1, 2, ... \}, \mathbb{N}_0 = \{0, 1, 2, ... \}$, Rechenregeln: $+$, $-$ bekannt
\begin{subdefinitionbox}
	Eine natürliche Zahl $b \in \mathbb{N}$ teilt eine natürliche Zahl $a \in \mathbb{N}$ (in Zeichen $b \mid a$), wenn es eine natürliche Zahl $c \in \mathbb{N}$ gibt, mit $a = b \times c$
\end{subdefinitionbox}
\begin{subdefinitionbox}
	\label{subdefgeradezahl}
	Eine Zahl $a \in \mathbb{N}$ heißt \underline{gerade}, falls $2\mid a$ gilt, d.h. falls es ein $c \in \mathbb{N}$ mmit $a = 2 \times c$ gibt.\\
	Eine Zahl $q \in \mathbb{N}$ heißt \underline{ungerade}, falls $q$ nicht gerade ist.
\end{subdefinitionbox}
\begin{subconjecturebox}
	$18$ ist gerade.
	\begin{subproofbox}
		$18 \in \mathbb{N}$, also können wir obrige Definition \ref{subdefgeradezahl} anwenden.\\
		Setze $c \coloneqq 9$\\
		Dann gilt $c \in \mathbb{N}$ und $18 = 2 \times 9$\\
		Also gilt $2 \mid 18$. Also ist $18$ gerade, $\square$
	\end{subproofbox}
\end{subconjecturebox}


%%%%%%%%%%%%%%%%%%%%%%%%%%%%%%%%%
%%% Apendix B von Junk/Traude %%%
%%%%%%%%%%%%%%%%%%%%%%%%%%%%%%%%%

\section{Apendix B von Junk/Traude}
\begin{definitionbox}
	\fbox{B.10} Definitionen
	\begin{enumerate}[label=]
		\item Nachweistext: Definition $\bigstar$ wird \fbox{Blubb} genannt, falls \circled{...} gilt.\\
			Schreibe: Zu zeigen \circled{...}
		\item Benutzungsbedingung: um zu benutzen, dass ein Objekt nach Definition $\bigstar$ Blubb genannt wird, falls \circled{...} gilt.\\
			Schreibe: Nach Definition von ``Blubb'', folgt \circled{...}
	\end{enumerate}
	\fbox{B.7} Existenzaussagen
	\begin{enumerate}[label=]
		\item Nachweistext: zu beweisen:
			\[ \exists x \in X : \circled{\text{...}} \]
			Schreibe:
			\begin{enumerate}[label=]
				\item Setze $x \coloneqq \_ \_ \_$.
				\item zu zeigen: $ x \in X$ mit \circled{...}
			\end{enumerate}
		\item Benutzungstext: Es gilt
	\end{enumerate}
	\begin{examplebox}
		Vor. Sei $a \in \mathbb{N}$\\
		Bew. Dann ist $2 \times a$ gerade\par
		Es sei $a \in \mathbb{N}$. Es gilt $2 \times a \in \mathbb{N}$. Nach Definition \ref{subdefgeradezahl} zu geraden Zahlen ist zu zeigen:
		\[ 2 \mid a \]
		d.h. zu zeigen: $ \exists c \in \mathbb{N} : s \times a = 2 \times c $\\
		Setzte $ c \coloneqq a $\\
		Dann gilt $ c \in \mathbb{N} $ und
		\[ 2 \times a = 2 \times c, \square \]
	\end{examplebox}
\end{definitionbox}
\begin{definitionbox}[\fbox{B.6} Allaussagen]
	Nachweistext: zu zeigen
	\[ \forall x \in X : \text{\circled{...}} \]
	\begin{indentpar}
		\indent Schreibe:\\
		Sei ein $ x \in X $ gegeben.\\
		Zeige: \circled{...}
	\end{indentpar}
	Benutzungstext:
	\begin{equation}
		\label{B.6.1}
		\forall x \in X : \text{\circled{...}} \text anzuwenden
	\end{equation}
	Dann muss ein $ a \in X $ vorliegen\\
	Wegen $ a \in X $ folgt (aus der $\forall$-Aussage)\\
	\indent\circled{...}
\end{definitionbox}
\begin{exercisebox}
	Zeige, dass die Summe zweier geraden natürlichen Zahlen auch gerade ist\par
	Vor.: Gegeben seien $ a \in \mathbb{N} $ und $ b \in \mathbb{N} $ gerade.\\
	Setzte
	\[ c \coloneqq a + b \in \mathbb{N} \]
	Beh.: $c$ ist gerade\\
	\begin{proofbox}
		Da $ c \in \mathbb{N} $, kann man Definition \ref{subdefgeradezahl} zu geraden Zahlen anwenden. Zu zeigen ist
		\[ 2 \mid c \]
		d.h. zu zeigen: $ \exists k \in \mathbb{N} : c = 2 \times k $\par
		Da $a, b$ gerade, gilt nach Definition \ref{subdefgeradezahl} zu geraden Zahlen
		\begin{equation}
			\label{exampleone}
			a = 2 \times m \text{ und}
		\end{equation}
		\begin{equation}
			\label{exampletwo}
			b = 2 \times n,
		\end{equation}
		mit $ m, n \in \mathbb{N} $, sodass
		\begin{equation}
			\label{examplethree}
			c = a + b = 2 \times m + 2 \times n = 2 \times ( m + n ),
		\end{equation}
		wobei wir für die erste Gleicheit \ref{exampleone} und \ref{exampletwo}, für die zweite Gleichheit die Rechenregeln für natürliche Zahlen benutzt haben.\\
		Setze
		\[ k \coloneqq m + n \]
		Dann gilt
		\[ k \in \mathbb{N} \]
		und wegen \ref{examplethree}
		\[ c = 2 \times ( m + n ) \]
		also
		\[
			c = 2 \times k, \tag*{$\square$}
		\]
	\end{proofbox}
\end{exercisebox}

\end{document}
