\documentclass[sectionformat = exercise]{gadsescript}

\setsemester{Winter Semester 2023/2024}%
\setuniversity{University of Konstanz}%
\setfaculty{Faculty of Science\\(Computer and Information Science)}%

\begin{document}
\maketitle
\section{Analyzing an Algorithm}
\begin{enumerate}[label=\alph*)]
	\item A suitable loop invariant $ \mathcal{INV}  $ is: 
		\begin{enumerate}[label=(\roman*)]
			\item $ A[i, \dotsc, n] $ is sorted non-descended
			\item $ \forall k,l \in \N : 1 \leq k < i \wedge i \leq j \leq n \implies A[k] \leq A[l] $
		\end{enumerate}
	\item Prove through Induction:
		\begin{description}
			\item[base case:] $ i = n $
				\begin{enumerate}[label=(\roman*)]
					\item $ A[i, \dotsc, n] = A[n, \dotsc, n] = A[n] $, an one-element array is always sorted
					\item $ \forall k,l \in \N : 1 \leq k < i \wedge i \leq  l \leq n: A[k] \leq A[l] $\\
						$ \iff \forall k, l \in \N : 1 \leq k < n \wedge n \leq  j \leq n \implies A[k] \leq A[l] $. Let $ A[m] $ be the greatest element in $ A $, that means $ A[(m+1)-1] \geq A[m + 1] $, so it would be moved to the next index, after increasing $ i, j, m $, by one this would repeat until $ j=n $ and $ A[m] $ was moved to the last index.
				\end{enumerate}
			\item[induction step:] $ i \curvearrowright i - 1 $
				\begin{description}
					\item[induction hypothesis] ~
						\begin{enumerate}[label=(\roman*)]
							\item $ A[i, \dotsc, n] $ is sorted non-descended
							\item $ \forall k,l \in \N : 1 \leq k < i \wedge i \leq  j \leq n \implies A[k] \leq A[j] $
						\end{enumerate}
				\end{description}
				Because of the induction hypothesis we now that $ A[i, \dotsc, n] $ is sorted non-descending (i) and that $ A[i - 1] \leq A[i] $ (ii), so $ A[i-1, \dotsc, n] $ is sorted non-descending.
				Let $ A[m] $ be the greatest element in $ A[1, \dotsc, i] $, that means $ A[(m+1)-1] \geq A[m + 1] $, so it would be moved to the next index, after increasing $ i, j, m $, by one this would repeat until $ j=i $ and $ A[m] $ was moved to the index $ i $.
				So that $ \forall k \in \N : 1 \leq k < i : A[k] \leq A[i] $.
				And because we know that $ \forall k,l \in \N : 1 \leq k < i \wedge i \leq  j \leq n \implies A[k] \leq A[j] $, because of the induction hypothesis, we have now proven (i) and (ii).\qed
		\end{description}
	\item \begin{enumerate}[label=(\roman*)]
		\item We have proven, that $ \mathcal{INV}  $ is a loop invariant.
		\item In the beginning the Array is unsorted, but the sorted area is empty, so the $ \mathcal{INV} $ is true
		\item after the $ n-1 $th iteration the Array $ A[1, \dotsc, n] $ is sorted non-descending because the Array $ A[2, \dotsc, n] $ is sorted non-descending and the element $ A[1] $ is not greater than every element in the Array $ A[2, \dotsc, n] $.
		\item the loop terminates after $ n-1 $ iterations
	\end{enumerate}
\end{enumerate}
	
\section{Hoar-Logic - Analysis}
\begin{enumerate}[label=\alph*)]
	\item $ x - 2 < 0 \iff x < 2 $ 
	\item $ z - 5 > 5 \iff z > 10 $
\end{enumerate}

\section{Dynamic Programming - Knapsack}
\begin{tabular}{|l|c|c|c|c|c|c|c|}
	\hline
	B 	& 0 & 1 & 2 & 3 & 4 & 5 & 6 \\\hline
	\hline
	$ I_1 $ & 0 & 2 & 4 & 6 & 7 & 9 & 11 \\\hline
	$ I_2 $ & 0 & 2 & 4 & 6 & 7 & 9 & 11 \\\hline
	$ I_3 $ & 0 & 1 & 4 & 5 & 7 & 9 & 10 \\\hline
	$ I_4 $ & 0 & 1 & 4 & 5 & 7 & 9 & 10 \\\hline
	\textit{using only other items}
		& 0 & 1 & 4 & 5 & 7 & 8 & 9 \\\hline

	
\end{tabular}


\end{document}

