\documentclass[sectionformat = aufgabe]{gadsescript}

\defaulttitleh
\settitle{Übungsblatt Nr. 3}
\setsubtitle{Jörg und Elias}
\setfaculty{Faculty of Science (Physics)}

\begin{document}
\maketitle

\section{Begleitendes Dreibein}
\begin{enumerate}[label=\alph*)]
	\item $ s(t) = \int_0^t |\vec v(T) | dT = \int_0^t | \dot{\vec r} (T) | dT = \int_0^t \left| \left( -v_{0,r} \sin(\omega_cT),v_{0,r}\cos(\omega_cT), v_{0,z} \right) \right| dT = \int_0^t \sqrt{ v_{0,r}^2 + v_{0,z}^2} dT = \sqrt{v_{0,r}^2 + v_{0,z}^2} t$\\
		Also $ t(s) = \frac{s}{\sqrt{v_{0,r}^2 + v_{0,z}^2}} $, folglich:
		\[ \vec r(s) = \left( \frac{v_{0,r}}{\omega_c} \cos\left(\frac{\omega_c}{\sqrt{v_{0,r}^2 + v_{0,z}^2}} s \right), \frac{v_{0,r}}{\omega_c} \sin\left(\frac{\omega_c}{\sqrt{ v_{0,r}^2 + v_{0,z}}} s \right), \frac{v_{0,z}}{\sqrt{v_{0,r}^2 + v_{0,z}^2}} s \right) \]
	\item \[ \vec T = \frac{d\vec r}{ds} = \left( -\frac{v_{0,r}}{\sqrt{v_{0,r}^2 + v_{0,z}^2}} \sin\left(\frac{\omega_c}{\sqrt{v_{0,r}^2 + v_{0,z}^2}} s \right), \frac{v_{0,r}}{\sqrt{v_{0,r}^2 + v_{0,z}^2}} \cos\left(\frac{\omega_c}{\sqrt{ v_{0,r}^2 + v_{0,z}}} s \right), \frac{v_{0,z}}{\sqrt{v_{0,r}^2 + v_{0,z}^2}} \right) \]
		\begin{align*}
			\vec N &= \frac{\frac{dT}{ds}}{\left| \frac{dT}{ds}\right|}\\
			 ~&= \frac{ \left(%
			 -\frac{v_{0,r} \omega_c}{{v_{0,r}^2 + v_{0,z}^2}} \cos\left(\frac{\omega_c}{\sqrt{v_{0,r}^2 + v_{0,z}^2}} s \right),%
			 -\frac{v_{0,r} \omega_c}{{v_{0,r}^2 + v_{0,z}^2}} \sin\left(\frac{\omega_c}{\sqrt{ v_{0,r}^2 + v_{0,z}^2}} s \right),%
			 0 \right)}%
			 {\sqrt{\frac{v_{0,r}^2\omega_c^2}{(v_{0,r}^2 + v_{0,z}^2)^2}}}\\
			 ~&= \frac{ \left(%
			 -\frac{v_{0,r} \omega_c}{{v_{0,r}^2 + v_{0,z}^2}} \cos\left(\frac{\omega_c}{\sqrt{v_{0,r}^2 + v_{0,z}^2}} s \right),%
			 -\frac{v_{0,r} \omega_c}{{v_{0,r}^2 + v_{0,z}^2}} \sin\left(\frac{\omega_c}{\sqrt{ v_{0,r}^2 + v_{0,z}^2}} s \right),%
			 0 \right)}%
			 {\frac{v_{0,r}\omega_c}{v_{0,r}^2 + v_{0,z}^2}}\\
			 ~&= \left(%
			 -\cos\left(\frac{\omega_c}{\sqrt{v_{0,r}^2 + v_{0,z}^2}} s \right),%
			 -\sin\left(\frac{\omega_c}{\sqrt{ v_{0,r}^2 + v_{0,z}^2}} s \right),%
			 0 \right)%
		\end{align*}
		\begin{align*}
			\vec B &= \vec T \times \vec N \\
			~&=  \Biggl(%
			\frac{v_{0,z}}{\sqrt{v_{0,r}^2 + v_{0,z}^2}} \sin\left( \frac{\omega_c}{\sqrt{v_{0,r}^2 + v_{0,z}^2}} s \right),\\
			~&\phantom{=}%
			- \frac{v_{0,z}}{\sqrt{v_{0,r}^2 + v_{0,z}^2}} \cos\left( \frac{\omega_c}{\sqrt{v_{0,r}^2 + v_{0,z}^2}} s \right),\\
			~&\phantom{=}%
			\frac{v_{0,z}}{\sqrt{v_{0,r}^2 + v_{0,z}^2}} \sin^2\left( \frac{\omega_c}{\sqrt{v_{0,r}^2 + v_{0,z}^2}} s \right) -%
			\left[ - \frac{v_{0,z}}{\sqrt{v_{0,r}^2 + v_{0,z}^2}} \cos^2\left( \frac{\omega_c}{\sqrt{v_{0,r}^2 + v_{0,z}^2}} s \right) \right]\Biggr)\\
			~&=  \left(%
			\frac{v_{0,z}}{\sqrt{v_{0,r}^2 + v_{0,z}^2}} \sin\left( \frac{\omega_c}{\sqrt{v_{0,r}^2 + v_{0,z}^2}} s \right),%
			- \frac{v_{0,z}}{\sqrt{v_{0,r}^2 + v_{0,z}^2}} \cos\left( \frac{\omega_c}{\sqrt{v_{0,r}^2 + v_{0,z}^2}} s \right),%
			1 \right)%
		\end{align*}
\end{enumerate}

\section{Reibungsprobleme}
\begin{enumerate}[label=\alph*)]
	\item $ F_{R,max} = \mu \cdot F_{N}$, also
		\begin{align*}
			(F_{H} + F_{E,max}) &= \mu \cdot F_{N}\\
			mg\sin\alpha + F_{E, max} &= \frac{5}{8} \cdot mg\cos\alpha\\
			F_{E, max} &= \frac{5}{8} \cdot mg\cos\alpha - mg\sin\alpha\\
			F_{E, max} &\approx \qty{36,4e6}{\kilogram\metre\per\square\metre}
		\end{align*}
	\item $ F_{R,max} = \mu_G \cdot F_{N} $, also $ F_G\sin\alpha = 0.3 \cdot F_G\cos\alpha $, d.h. $ \tan\alpha = 0.3 \iff \alpha = \arctan 0.3 \approx \num{.29}$
		Die Höhe des Kegels über dem Silo ist also $ \frac{d}{2}\sin\arctan 0.3 $ und das Volumen insgesamt ist:
		\[ \frac{1}{3} \pi \cdot \left(\frac{d}{2}\right)^2 \cdot \frac{d}{2} \sin\arctan 0.3 + \pi \cdot \left(\frac{d}{2}\right)^2\cdot h = \pi \cdot \left(\frac{d}{2}\right)^2 \left( \frac{d}{6}\sin\arctan 0.3 + h \right) \approx \qty{2393.8}{\cubic\metre}\]
\end{enumerate}
\end{document}
