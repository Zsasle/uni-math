\documentclass[sectionformat = aufgabe]{gadsescript}

\settitle{Übungsblatt Nr. 2}

\begin{document}
\maketitle

\section{Orthonomalbasis}
\begin{enumerate}[label=\alph*)]
	\item 
		\begin{alignat*}{3}
			&\left( \vec e_1 + \vec e_2 \right) \cdot \vec e_3
				&\omit\hfil$\overset{\text{Distr.}}{=}$\hfil&\vec e_1 \cdot \vec e_3 + \vec e_2 \cdot \vec e_3\\
			&~&\omit\hfil$=$\hfil&\delta_{13} + \delta_{23}\\
			&~&\omit\hfil$=$\hfil& 0 + 0\\
			&~&\omit\hfil$=$\hfil&0
		\end{alignat*}
		\begin{alignat*}{3}
			&\left(4\vec e_1 + 3 \vec e_2 \right) \cdot \left(7\vec e_1 - 16\vec e_3\right)
				&\omit\hfil$\overset{\text{Distr.}}{=}$\hfil&\left( 4\vec e_1 + 3\vec e_2 \right) \cdot 7\vec e_1 - \left( 4\vec e_1 + 3\vec e_2 \right) \cdot 16\vec e_3\\
			&~&\omit\hfil$\overset{\text{Distr.}}{=}$\hfil&28\vec e_1\vec e_1 + 21\vec e_1 \vec e_2 - 64\vec e_1\vec e_3 - 48\vec e_2 \vec e_3\\
			&~&\omit\hfil$=$\hfil&28\delta_{11} + 21\delta_{12} - 64\delta_{13} - 48\delta_{23}\\
			&~&\omit\hfil$=$\hfil&28
		\end{alignat*}
	\item Genau dann wenn zwei Vektoren orthogonal sind, oder wenn einer der Vektoren der Nullvektor is, dann ist das Skalarprodukt zweier Vektoren 0. Also wollen wir ein $X$ finden, wofür gilt:
		\[ \vec a \cdot \vec b = \left(2\vec e_1 - 5\vec e_2 + X\vec e_3 \right) \cdot \left( -\vec e_1 + 2\vec e_2 - 3\vec e_3 \right) = 0 \]
		\begin{align*}
			\left(2\vec e_1 - 5\vec e_2 + X\vec e_3 \right) \cdot \left( -\vec e_1 + 2\vec e_2 - 3\vec e_3 \right) &= 0\\
			\sum_{i = 0}^{3}a_ib_i &= 0\\
			2\cdot(-1) + (-5)\cdot2 + X\cdot(-3) &= 0\\
			-2 -10 -3X &= 0\\
			3X &= 12\\
			X &= 4\\
		\end{align*}
		Also für $ X \coloneqq 4 $ sind die Vektoren $\vec a $ und $ \vec b $ parallel, oder einer der beiden ist der Nullvektor. Da aber $ \vec a \neq \vec0 $ und $ \vec b \neq \vec0 $ sind die Vektoren parallel
	\item Genau dann, wenn die Vektoren $ \vec v $ und $ \vec w $ linear unabhängig sind, gilt:
		\[ \forall \lambda_1,\lambda_2 \in \R\setminus\{0\} : \lambda_1\vec v + \lambda_2\vec w \neq 0 \]
		Seien $ \lambda_1, \lambda_2 \in \R\setminus\{0\} $ gegeben, zu zeigen:\\
		$\lambda_1 \vec v + \lambda_2 \vec w \neq \vec 0 $, wir führen einen Beweis durch Widerspruch und nehmen dazu an $ \exists\lambda_1,\lambda_2\in\R\setminus\{0\}:\lambda_1\vec v+\lambda_2\vec w = 0$, dann gilt:
		\begin{align*}
			\lambda_1v_1 + \lambda_2w_1 &= 0,\\
			\lambda_1v_2 + \lambda_2w_2 &= 0 \text{ und}\\
			\lambda_1v_3 + \lambda_2w_3 &= 0.
		\end{align*}
		Also gilt für die erste Gleichung:
		\begin{align*}
			\lambda_1\cdot v_1 + \lambda_2w_1 &= 0\\
			\lambda_1 \cdot 1 &= -\lambda_2\cdot (-3)\\
			\lambda_1 &= 3\lambda_2.
		\end{align*}
		Dies in die zweite Gleichung eingesetzt ergibt:
		\begin{align*}
			\lambda_1v_2 + \lambda_2w_2 &= 0 \\
			3\lambda_2v_2 + \lambda_2w_2 &= 0 \\
			\lambda_2\left(3v_2 + w_2\right) &= 0 \\
			\lambda_2\left(3 \cdot (-1) + 2\right) &= 0 \\
			\lambda_2\left(-1\right) &= 0 \\
			\lambda_2 &= 0
		\end{align*}
		Da aber $ \lambda_2 \neq 0 $ führt dies zu einem Widerspruch und die die Annahme $ \exists\lambda_1,\lambda_2\in\R\setminus\{0\}:\lambda_1\vec v+\lambda_2\vec w = 0$, war falsch, also gilt $\forall \lambda_1,\lambda_2 \in \R\setminus\{0\} : \lambda_1\vec v + \lambda_2\vec w \neq 0 $.\qed
\end{enumerate}

\section{Raketengleichung}
\begin{enumerate}[label=\alph*)]
	\item Aus der Vorlesung wissen wir, dass gilt im Allgemeinen:\\
		\[ v(T) = - v_{rel} \cdot \ln{\frac{m_T}{m_0}} - gT \]
		da die Rakete pro Zeiteinheit die Gasmenge $\alpha$ mit der Geschwindigkeit $v_0$ ausstößt und die Anfangsmasse $m_0$ hat, gilt für $m_T$:
		\[ m_T = m_0 - \alpha t, \]somit gilt:
		\[ v(T) = v_0 \cdot \ln(\frac{m_0 - \alpha T}{m_0}) - gT \]
		für $T$, die Brenndauer.
	\item $s(t) = \int_{t_0}^{t} v(T) dT$:
		setze $ u \coloneqq \frac{m_0 - \alpha T}{m_0} $, dann gilt $ du = - \frac{\alpha dT}{m_0} $
		\begin{align*}
			v(T) &= - v_0\ln(\frac{m_0 - \alpha T}{m_0}) - gT\\
			v(T)dT &= - v_0\ln(\frac{m_0 - \alpha T}{m_0})dT - gTdT\\
			v(T)dT &= - v_0\ln(u)\frac{du}{du}dT - gTdT\\
			v(T)dT &= - v_0\ln(u)\frac{du}{-\frac{\alpha dT}{m_0}}dT - gTdT\\
			v(T)dT &= v_0\ln(u)\frac{m_0du}{\alpha} - gTdT\\
			v(T)dT &= \frac{v_0m_0}{\alpha}\ln(u)du - gTdT\\
			\int_0^t v(T)dT &= \frac{v_0m_0}{\alpha}\int_{u_0}^{u_t}\ln(u)du - \int_0^t gTdT\\
			s(t) &= \frac{v_0m_0}{\alpha}\left[u_t(\ln(u_t) - 1) - u_0(\ln(u_0) - 1)\right] - \frac{1}{2}gt^2 + C
		\end{align*}
		setzte für $u_t \coloneqq \frac{m_0 - \alpha t}{m_0}$ und für $u_0 \coloneqq \frac{m_0 - \alpha 0}{m_0} = 1$

		\begin{align*}
			s(t) &= \frac{v_0m_0}{\alpha}\left[\frac{m_0 - \alpha t}{m_0}\left(\ln\left(\frac{m_0 - \alpha t}{m_0}\right) - 1\right) - (\ln(1) - 1)\right] - \frac{1}{2}gt^2 + C\\
			s(t) &= \frac{v_0m_0}{\alpha}\left[\frac{m_0 - \alpha t}{m_0}\left(\ln\left(\frac{m_0 - \alpha t}{m_0}\right) - 1\right) + 1\right] - \frac{1}{2}gt^2 + C\\
			s(t) &= \frac{v_0m_0}{\alpha}\left[\left( 1 - \frac{\alpha t}{m_0}\right)\left(\ln\left(\frac{m_0 - \alpha t}{m_0}\right) - 1\right) + 1\right] - \frac{1}{2}gt^2 + C\\
			s(t) &= \frac{v_0m_0}{\alpha}\left[ \ln\left(\frac{m_0 - \alpha t}{m_0}\right) - 1 - \frac{\alpha t}{m_0}\left(\ln\left(\frac{m_0 - \alpha t}{m_0}\right) - 1\right) + 1\right] - \frac{1}{2}gt^2 + C\\
			s(t) &= \frac{v_0m_0}{\alpha}%
				\left[%
					\ln\left(\frac{m_0 - \alpha t}{m_0}\right)%
					- \frac{\alpha t}{m_0}%
					\left(\ln\left(\frac{m_0 - \alpha t}{m_0}\right) - 1\right)%
				\right] - \frac{1}{2}gt^2 + C\\
			s(t) &= \frac{v_0m_0}{\alpha}%
				\left[%
					\left( 1 - \frac{\alpha t}{m_0} \right)%
					\ln\left(\frac{m_0 - \alpha t}{m_0}\right)%
					 + \frac{\alpha t}{m_0}%
				\right] - \frac{1}{2}gt^2 + C\\
			s(t) &= \frac{v_0}{\alpha}%
				\left[%
					\left( m_0 - \alpha t\right)%
					\ln\left(\frac{m_0 - \alpha t}{m_0}\right)%
				\right]%
				+ v_0 t%
				- \frac{1}{2}gt^2 + C\\
			s(t) &= \frac{v_0m_0}{\alpha}%
				\left(%
					\ln\left(\frac{m_0 - \alpha t}{m_0}\right)%
				\right)%
				+ v_0\left(1 - \ln\left(\frac{m_0 - \alpha t}{m_0}\right)\right) t%
				- \frac{1}{2}gt^2 + C\\
			s(t) &= %
				v_0\left(1 - \ln\left(\frac{m_0 - \alpha t}{m_0}\right)\right) t%
				- \frac{1}{2}gt^2 + s_0
		\end{align*}
	\item Bei mehrstufigen Rakten wird Balast abgeworfen, wodurch das Gewicht verringert wird. Bei geringerem Gewicht muss nach $ F = m\cdot a \implies \frac{1}{m} \propto m $ bei gleicher Kraft, also bei gleicher Schubkraft.
\end{enumerate}

\section{Freier Fall}
\begin{enumerate}[label=\alph*)]
	\item 
		\begin{align*}
			\vec r(t) &= \int_{t_0}^{t} \vec v dt + h_0\\
			\vec r(t) &= \int_{t_0}^{t} \left(\int_{t_0}^{t} \vec a dt + \vec v \right) dt + h_0\\
			\vec r(t) &= \int_{t_0}^{t} \int_{t_0}^{t} \vec a dt^2 + \int_{t_0}^{t} \vec v dt + h_0\\
		\end{align*}
		Da wir annehmen, dass wir nicht nach oben oder unten springen gilt nach dem Superpositionsprinzip:
			\[h(t) = -\frac{1}{2}gt^2 + h_0\]
		Dabei ist $h_0 = \qty{8}{\metre}$. Um die Auftreffzeit zu berechnen, müssen wir $ h(t) = 0 $ setzten:
		\begin{align*}
			h(t) &= -\frac{1}{2}gt^2 + h_0\\
			\qty{0}{\metre} &= -\frac{1}{2}gt^2 + \qty{8}{\metre}
		\end{align*}
		dann gilt nach der Binomischen Formel:
		\begin{align*}
			t_{1,2} &= \frac{-0 \pm \sqrt{0^2 + 2g\cdot \qty{8}{\metre}}}{2\cdot \left(-\frac{1}{2}g\right)}\\
			t_{1,2} &= \frac{\mp \sqrt{g \cdot \qty{16}{\metre}}}{g}\\
			t_{1,2} &= \frac{\mp 4\sqrt{\qty{10}{\square\metre\per\square\second}}}{\qty{10}{\metre\per\square\second}}
		\end{align*}
		\[ t_1 = \frac{- 4\sqrt{10}\,\unit{\metre\per\second}}{\qty{10}{\metre\per\square\second}} \]
		\[ t_2 = \frac{4\sqrt{10}\,\unit{\metre\per\second}}{\qty{10}{\metre\per\square\second}} \]
		Da $ t_1 $ eine negative Zeit ist, wir aber erst zur Zeit $ \qty{0}{\second} $ losspringen, kann dies nicht sein und die Auftreffzeit ist $ t_2 = \frac{4\sqrt{10}\,\unit{\metre\per\second}}{\qty{10}{\metre\per\square\second}} = \frac{2\sqrt{5}}{5}\unit{\second}$
	\item Es gilt weiterhin
		\begin{align*}
			\vec r(t) &= \int_{t_0}^{t} \int_{t_0}^{t} \vec a dt^2 + \int_{t_0}^{t} \vec v dt + h_0\\
		\end{align*}
		und $ h_0 = \qty{8}{\metre} $, da für die Höhe nach dem Superpositionsprinip nur die Geschwindigkeit in die y-Richtung von Bedeutung ist, gilt für $ t_0 = \qty{0.5}{\second} $
		\begin{align*}
			h(t) &= \int_{t_0}^{t} \int_{t_0}^{t} \vec a dt^2 + \int_{t_0}^{t} v_y dt + h_0\\
			h(t) &= - g(t- t_0)^2 + v_y \cdot (t-t_0) + \qty{8}{\metre}\\
		\end{align*}
		Dabei soll $ h(t_2) = \qty{0}{\metre} $, also:
		\[\qty{0}{\metre}  = -g\left(\frac{2\sqrt{5}}{5}\unit{\second} - \qty{0.5}{\second} \right)^2 + v_0 \left(\frac{2\sqrt{5}}{5}\unit{\second} - \qty{0.5}{\second} \right)  + \qty{8}{\metre}\]
		\[ \qty{10}{\metre\per\square\second}\cdot\left(%
			\frac{4\cdot5}{25}\unit{\square\second} - 2\cdot\frac{2\sqrt{5}}{5}\cdot\qty{0.5}{\square\second} + \qty{0.25}{\square\second} \right)%
			- \qty{8}{\metre} = v_0 \frac{2\sqrt{5}}{5}\unit{\second}\]
		\[ \qty{10}{\metre}\cdot\left(\frac{4}{5} - \frac{2\sqrt{5}}{5} + 0.25 \right)-  \qty{8}{\metre} = v_0 \frac{2\sqrt{5}}{5}\unit{\second}\]
		\[ \qty{8}{\metre} - \frac{4\sqrt{5}}\,\unit{\metre} + 2.5\,\unit{\metre} -  \qty{8}{\metre} = v_0 \frac{2\sqrt{5}}{5}\unit{\second}\]
\end{enumerate}

\end{document}
