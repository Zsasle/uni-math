\documentclass{gadsescript}

\usepackage[ngerman]{babel}
%\usepackage[english=quotes]{csquotes}


\settitle{Integrierter Kurs I}

\begin{document}
\maketitle
\begin{itemize}
	\item novak.uni-konstanz.de
	\item Tobias Dannegger macht Übungsgruppe Zeug
	\item nach der Vorlesung in Übungsgruppen einteilen
		\begin{itemize}
			\item Übungsgruppe wechseln möglich
		\end{itemize}
	\item theoretisch erste Aufgaben heute, aber nicht weil nächste Woche Mittwoch Feiertag
	\item Bepunktete Aufgaben
		\begin{itemize}
			\item ist abzugeben (und zu machen lol)
			\item wird bepunkted
		\end{itemize}
	\item Unbepunktete Aufgaben
		\begin{itemize}
			\item sagen ob wir sie gemacht haben oder nicht
			\item[->] an der Tafel dann vorrechnen
		\end{itemize}
	\item Präsenzaufgaben
		\begin{itemize}
			\item werden nicht bewertet und sind zur Selbstkontrolle
		\end{itemize}
	\item Scripts on ILIAS
		\begin{itemize}
			\item hat er während Corona geschrieben
			\item ``aber trotzdem schöne Ergänzung''
		\end{itemize}
\end{itemize}

\section{Inhalt}
\%\%\% Wo finden? aber gibt Dokument \%\%\%
\begin{itemize}
	\item  Messung und Einheiten
	\item Vektoralgebra und physikalischer Raum
	\item Mechanik des Massenpunkts
	\item Eindimensionale Systeme
	\item Vektoranalysis
	\item Bewegung in drei Dimensionen
	\item Erhaltungssätze in Mehrteilchensystemen
	\item Dynamik starrer ausgedehnter Körper
\end{itemize}
\%\%\% Fertig \%\%\%

Vorlesungen werden \textbf{nicht} aufgezeichnet

\section{Willkommen zur Vorlesung Integrierter Kurs Physik 1: ``Mechanik''}
Prof. Dr. Marina Müller\\
Vorlesungs-``Skript'': Wolfgang Demtröder, Experimentalphysik 1 Mechanik und Wärme\\
\indent\quad weiter Materialien auf ILIAS\\
\indent\qquad V01\_2023-10-23\_Folien.pdf\\

\textbf{lesen Sie Lehrbücher}
\begin{itemize}
	\item Brandt, Dahmen: Mechanik
	\item Torsten Fließbach: Mechanik
	\item Friedhelm Kuypers: Klassische Mechanik
	\item Dieter Meschede Hrsg.: Gerthsen Physik\\
		\indent$\quad \vdots$
\end{itemize}

\textbf{gehen Sie in die Uni, wenn Sie recherchieren wollen}

\subsection{Womit beschäftigt sich die Physik?}
\begin{itemize}
	\item Physik, von griechisch $\Phi\nu\sigma\iota\kappa...$\\
	\item Die Physik beschäftigt sich vom kleinsten im Universum bis zum größten
	\item Wir beginnen mit der Mechanik
		\begin{itemize}
			\item eines der ältesten Teilgebiete der Physik
			\item Grundlage
		\end{itemize}
\end{itemize}

\begin{itemize}
	\item Daten
		\begin{itemize}
			\item sind eindeutig
		\end{itemize}
	\item Fakten
		\begin{itemize}
			\item sind nicht eindeutig
			\item Modell/Interpretation liegt zugrunde
		\end{itemize}
\end{itemize}

``... there are known knowns;\\there are things we know we know.\\~\\We also know there are known unkowns; that to say...''\\
Wir beschäftigen uns in den Vorlesungen mit dem ``known knowns''\\
\indent\quad KI kann nur \textbf{``known knowns''}-Zeug, weil input nur aus \textbf{``known knowns''} besteht??

\section{Integreiter Kurs Physik 1}
\subsection{Messungen und Einheiten [D1, 1.6]}
Für experminetelle Beobachtungen brachen wir eindeutige Meßvorschriften:\\
Physikalische Größe:
\begin{align*}
	u &= \{u\}[u]\\
	~&= \{u\} \text{ Zahlenwert, Meßzahl}\\
	~&= [u]: \text{ Einheit}
\end{align*}

\textbf{Wichtige Grundgrößen in der Physik}
\begin{itemize}
	\item Zeit $t$ in Sekunden: $[t] = \unit{\second}$
	\item Länge $L$ in Metern: $[L] = \unit{\metre}$
	\item Masse $m$ in Kilogramm: $[m] = \unit{\kilogram}$
	\item SI: systéme internationale (diumités)
\end{itemize}
Alle Basiseinheiten sind Normale oder Standards definiert\\
\indent$\quad\Rightarrow$ seit 2019: Definition der Einheiten über \textbf{Naturkonstanzen} = const.\\

\textbf{Mechanik}	\quad Zeit		\quad Länge	\quad Masse\\
\indent$\qquad\qquad\qquad\unit{\second}$\\
\textbf{Thermodynamik}	\quad Temperatur	\quad Stoffmenge\\
\textbf{Elektrodynamik}	\quad Stromstärke\\
\textbf{Optik}		\quad Lichtstärke

\subsubsection{Zeit [D1, 1.6 / Wikipedia (Wiki)]}
Allgemein: jedes sich periodisch wiederholende Phänomen\\
z.B. Natur gibt Tag bzw Jahr $\rightarrow$ in der Praxis zu groß $\rightarrow$ teilen in Std, Min, s\\
\indent\quad 1 Tag $\leftrightarrow$ Rotationsperoder der Erde $\hat= 24 \times 60 \times 60$\\

\textbf{SI System:}\\
``Die Sekunde ist die Dauer von $9192.631.770$ Schwingungsperioden der Strahlung, die dem Übergang zwischen den beiden Hyperfeinstrukturniveaus des Grundzustand eines ruhenden Atoms des Isotops Cäsiums-133 entspricht.''\\

\begin{tikzpicture}
	\draw (0, 0) -- (4, 0);
	\draw (0, 2) -- (4, 2);
	\draw[<->] (0.2, 0) -- node [anchor = west] {$\Delta E$}(0.2, 2);
\end{tikzpicture}
\[ \Delta E = h \Delta \nu_{Cs} \rightarrow \Delta\nu_{Cs} = \frac{\Delta E}{h} \]
\[ \Delta\nu \coloneqq 9192631700 \unit{\per\second} \]
\[ \Delta\nu_{Cs} = 9.192631770 \unit{\giga\hertz} \]
\[ \Delta\nu = 9.19... \cdot 10^{9}\unit{\hertz} \]

\subsubsection{Länge}
lichtgeschwindigkeit im Vakuum $ c_0 \coloneqq 299792458 \unit{\metre\per\second} $\\
$\qty{1}{\metre} \hat= \text{ Entfernung, die Licht ``$\frac{1}{c} \unit{\second}$'' zurücklegt} $\\

\subsubsection{Masse}
Planksche Wirkungsquantum (``Energie $\times$ Zeit'')
\[ h = \qty{6.62607015e-34}{\kilogram\square\meter\per\second} \]
$\rightarrow$ Masse \textbf{formal} exakt definiert über Definition von Sekunde und Meter\\

\textbf{MKS System:}
\begin{itemize}
	\item Sekund $\rightarrow$ $\Delta\nu_{Cs}$
	\item Meter $\rightarrow$ $ c[\unit{\metre\per\second]}$  $\rightarrow$(impplizit) $\unit{s}$
	\item Kilogram $\rightarrow$ $h [\unit{\kilogram\square\metre\per\second]}$ $\rightarrow$(impliziert) $\unit{s}$, $\unit{\metre}$
\end{itemize}


\end{document}
