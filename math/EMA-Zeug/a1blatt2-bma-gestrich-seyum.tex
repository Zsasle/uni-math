\documentclass{gadsescript}

\settitle{BMA 2}

\begin{document}
\maketitle
\begin{enumerate}[label=(\alph*)]
	\item Um zu zeigen, dass $ a \in A $ gilt zu zeigen: $ a \in \N $ und $  a^2 + 2a > 3 $\\
		Setze nun $ a \coloneqq 1000 $:\\
		$ 1000 \in \N $ und $ 1000^2 +2 \times 1000 = 1002000 > 3 $ also ist $ 1000 \in A $\\
		~\\
		Um zu zeigen, dass $ b \in B $ gilt zu zeigen: $ b \in \N $ und $  2 - \frac{2}{b} > -b $\\
		Setze nun $ b \coloneqq 4 $:\\
		$4 \in \N $ und $ 2 - \frac{3}{4} = \frac{2\times 4 - 3}{4} = \frac{8-3}{4} = \frac{5}{4} > -4 $, also $ 4 \in B $
	\item \label{item:b}Um zu zeigen, dass $ c \in C $ gilt zu zeigen: $ c \in \Z $ und $ \frac{2c}{5} < \frac{4}{c^2+1}  $\\
		Setze nun $ c \coloneqq 0 $:\\
		$ 0 \in \Z $ und $ \frac{2\times0}{5} = 0 < 4 = \frac{4}{0^2 + 1} $, also ist $ 0 $ in $ C $
	\item \label{item:c}Um zu zeigen, dass $ A \subset B $ ist zu zeigen, dass $ \forall a \in A: a \in B $:\\
		Sei $ a \in A $ zu zeigen $ a \in B $:\\
		$ a \in \N $ gegeben durch $ a \in A $ und $ a^2 + 2a > 3 $.\\
		Zu zeigen $ a \in B $. D.h. zu zeigen $ a \in \N $ und $  2 - \frac{2}{a} > -a $\\
		Da $ a \in \N $ gegeben, durch $ a \in \N $ bleibt zu zeigen $  2 - \frac{2}{a} > -a $\\
		Es gelte $ a^2 + 2a > 3 $, zu zeigen $  2 - \frac{2}{a} > -a $:
		Durch Termumformung folgt:
		\begin{alignat*}{3}
			a^2 + 2a &> 3&&\mid :a\\
			a + 2 &> \frac{3}{a} &&\mid -\frac{3}{a} - a\\
			2 - \frac{3}{a} &> -a &&~
		\end{alignat*}
		also gilt:
		\begin{equation}
			\label{eq:1}
			( a^2 + 2a > 3 \iff 2 - \frac{3}{a} > -a )
		\end{equation}
		Also ist $ \forall a \in A: a \in B $\qed
	\item Um zu zeigen $ A = B $ gilt zu zeigen $ A \subset B $ und $ B \subset A $.\\
		Mit \ref{item:c} ist $ A \subset B $ gezeigt und es bleibt $ B \subset A $ zu zeigen.\\
		Um zu zeigen, dass $ B \subset A $ ist zu zeigen, dass $ \forall b \in B: b \in A $:\\
		Sei $ b \in B $ zu zeigen $ b \in A $:\\
		$ b \in \N $ gegeben durch $ b \in B $ und $ 2 - \frac{3}{b} > -b $.\\
		Zu zeigen $ b \in A $. D.h. zu zeigen $ b \in \N $ und $ b^2 + 2b > 3 $\\
		Da $ b \in \N $ gegeben, durch $ b \in \N $ bleibt zu zeigen $ b^2 + 2b > 3$\\
		Aus (\ref{eq:1}) folgt, dass wenn $ 2 - \frac{3}{b} > -b $ auch $ b^2 + 2b > 3$\\
		Also ist $ \forall b \in B: b \in A $\qed
	\item Um zu zeigen, dass $ A = C $, gilt zu zeigen, dass $ C \subset A $ und $ A \subset C $.\\
		Zu zeigen $ C \subset A $:\\
		Da aus \ref{item:b} folgt, dass $ 0 \in C $, zu zeigen $ 0 \in A $, also gilt im insbesondere zu zeigen $ 0 \in \N $ aber $ 0 \notin \N $ ist $ 0 \notin A $ und $ C \nsubseteq A $, also $ C \neq A $\qed

\end{enumerate}

\end{document}
