\documentclass{gadsescript}

\settitle{BMA 4}

\begin{document}
\maketitle

\section*{Aufgabe 4: Beweismechanik}
\begin{enumerate}[label=\alph*)]
	\item \textbf{Vor.:}
		\[ A \coloneqq \{x \in \R : | x - 1 | \geq 2 \}, \]
		\[ B \coloneqq \{x \in \R : (x \leq 2) \wedge ( x^2 - 1 < 0) \} \]
		\textbf{Beh.:} $ A \subset \R \setminus B $
		\begin{proof*}
			Z.z $ \forall a \in A : a \in \R \setminus B $\\
			Gegeben $ a \in A $, d.h. $ a \in \R $ und $ | a - 1 | \geq 2 $, zu zeigen $ a \in \R \setminus B $.\\
			Also zu zeigen $ a \in \R $ und $ a \notin B $\\
			$ a \in \R $ gegeben, noch zu zeigen $ a \notin B $.
			Da $ | a - 1 | \geq 2 $ gegeben, gilt:\\
			\begin{description}
				\item[Fall 1:] $ a - 1 \geq 2 $, also $ a \geq 3 $.\\
					Wir führen einen Beweis durch Widerspruch und nehmen dazu an, $ a \in B $. Dann gilt insbesondere $ ( a \leq 2 ) \wedge ( a^2 - 1 < 0 ) $, also insbesondere $ a \leq 2 $, was im Widerspruch zu $ a \geq 3 $ steht. Also ist unsere Annahme falsch, dass $ a \in B $, folglich gilt $ a \notin B $.
				\item[Fall 2:] $ - ( a - 1 ) \geq 2 $, also $ -a  +1 \geq 2 $, also $ a \leq -1 $, also $ -a \geq -(-1) = 1 $
					\begin{align*}
						a &\leq -1\\
						a^2 &\geq -a \underbrace{\geq}_{\text{da $ -a \geq 1 $}} 1\\
						a^2 - 1 \geq 0
					\end{align*}
					Wir führen einen Beweis durch Widerspruch und nehmen dazu an, dass $ a \in B $. Dann gilt insbesondere $ ( a \leq 2 ) \wedge ( a^2 - 1 < 0 ) $, also insbesondere $ a^2 - 1 < 0 $, was in einem Widerspruch zu $ a^2 - 1 \geq 0 $ steht. Also war die Annahme falsch, dass $ a \in B $ und daraus folgt, dass $ a \notin B $ gilt. \qed
			\end{description}
		\end{proof*}
	\item \textbf{Vor.:} $ X $ eine Menge und $ A, B \subset X $ zwei Teilmengen von $ X $.\\
		\textbf{Beh.:}
		\[ X \setminus ( A \setminus B ) = ( X \setminus A ) \cup B \]
		\begin{proof*}
			Zu zeigen $ X \setminus ( A \setminus B ) \subset ( X \setminus A ) \cup B $ und $ X \setminus ( A \setminus B ) \supset ( X \setminus A ) \cup B $
			\begin{description}
				\item[`$\subset$':] zu zeigen $ \forall x \in X \setminus ( A \setminus B ) : x \in ( X\setminus A) \cup B $, sei $ x \in X \setminus ( A \setminus B ) $ gegeben, also $ x \in X $ und $ x \notin A \setminus B $,
					zu zeigen $ x \in ( X \setminus A ) \cup B $, also zu zeigen $ x \in ( X \setminus A ) \vee x \in B $, also zu zeigen $ \left( x \in X \wedge  x \notin A \right) \vee x \in B $.
					Da $ x \in X $ und $ x \notin A \wedge x \in B $ gegeben, gilt $ ( x \in X \wedge x \notin A ) \vee x \in B $, also $ x \in (X\setminus A) \cup B$
					\begin{align*}
						x &\notin \{ a \in X : a \in A \wedge a \notin B \}\\
						\neg &\left( x \in \{ a \in X : a \in A \wedge a \notin B \} \right)\\
						\neg &\left( x \in A \wedge x \notin B \right)\\
						x &\notin A \vee x \in B \\
						\left( w \wedge x \notin A \right) \vee x \in B
					\end{align*}
					Und da $ x \in X $ gegeben gilt: $ \left( x \in X \wedge x \notin A \right) \vee x \in B $, was zu zeigen war
				\item[`$\supset$'] Also zu zeigen $ \forall x \in ( X \setminus A ) \cup B: x \in X \setminus ( A \setminus B )$.\\
					Sei \begin{equation} \label{eq:1} x \in ( X \setminus A ) \cup B \end{equation} gegeben,
					zu zeigen $ x \in X \setminus ( A \setminus B ) $.\\
					Aus $ X \setminus A \subset X $ und $ B \subset X $ folgt $ x \in X $.
					Aus \eqref{eq:1} folgt:
					\[ x \in X \setminus A \vee x \in B \mid \text{Def.} \]
					\[ ( x \in X \wedge x \notin A ) \vee x \in B \mid \text{Distributivität}\]
					\[ ( x \in X \vee x \in B ) \wedge ( x \notin A \vee x \in B ) \mid \text{Da $ B \subset X $}\]
					\[ ( x \in X ) \wedge ( x \notin A \vee x \in B ) \mid \text{De Morgan}\]
					\[ ( x \in X ) \wedge ( \neg \left ( x \in A \wedge x \notin B \right)) \mid \text{Definition}\]
					\[ ( x \in X ) \wedge ( \neg \left ( x \in \{a \in A \wedge a \notin B \} \right) ) \]
					\[ ( x \in X ) \wedge x \notin \{a \in A \wedge a \notin B \} ) \mid \text{Definition}\]
					\[ ( x \in X ) \wedge ( x \notin A \setminus B ) \mid \text{Definition} \]
					\[ x \in X \setminus ( A \setminus B ) \qed \]
			\end{description}
		\end{proof*}
\end{enumerate}

\end{document}
