\documentclass{gadsescript}

\usepackage[ngerman]{babel}

\settitle{Zettel 1}
\begin{document}
\maketitle
\section*{Aufgabe 2}
$ n \coloneqq \text{ Anzahl der Studierenden }, n \in \N $\\
Beh.: Alle Studierende der Analysis 1-Vorlesung haben denselben Namen.\\
\begin{description}
	\item[I.A. ( $ n = 1 $)] ein*e Student*in hat einen Namen
	\item[I.S.]
		\begin{description}
			\item[I.V.] $ n $ Studenten haben denselben Namen
		\end{description}
		
\end{description}

\noindent\rule{\textwidth}{0.4pt}

Der Fehler liegt darin, dass für $ n = 1 $ die Mengen $ M_1 \coloneqq \{ a_1, \dotsc, a_n \} $ und $ M_2 \coloneqq \{ a_2, \dotsc, a_n+1 \} $, folgendes gilt: $M_1 \cap M_2 = \emptyset$, also gibt es keine Personen die sowohl in $ M_1 $ und $ M_2 $ sind. Dementsprechend gilt nicht, dass $ a_1 $ und $ a_2 $ denselben Namen haben müssen.

\section*{Aufgabe 3}
Wir wollen mithilfe vollständiger Induktion beweisen, dass
\[ \forall n \in \N : \sum_{i=1}^{n} i = \frac{ n ( n+1 ) }{ 2 }. \]
\begin{description}
	\item[I.A. ($ n = 1 $)]
		\[ \sum_{ i = 1 }^{ 1 } i = 1 = \frac{ 2 }{ 2} \frac{ 1 ( 1 + 1 ) }{ 2 } \]
	\item[I.S.] ~
		\begin{description}
			\item[Induktionsannahme ( = Induktionsvorraussetzung (I.V.))]
				\[\sum_{ i = 1 }^{ n } i = \frac{ n ( n + 1 }{ 2 } \]
		\end{description}
		z.z.
		\begin{equation}
			\label{eq:Aufgabe 3.zu-zeigen}
			\sum_{ i = 1 }^{ n + 1} i = \frac{ ( n + 1 ) ( ( n + 1 ) + 1 ) }{ 2 }
		\end{equation}
		Für die linke Seite gilt:
		\[ \sum_{ i = 1 }^{ n + 1 } i = \sum_{ i = 1 }^{ n } i + n + 1 \]
		nach I.V. und den Rechengesetzen für Addition und Multiplikation ganzer Zahlen gilt:
		\begin{align*}
			\sum_{ i = 1 }^{ n + 1 } i &= \sum_{ i = 1 }^{ n } i + n + 1\\
			~&= \frac{ n ( n + 1 ) }{ 2 } + n + 1 \\
			~&= \frac{ n ( n + 1 ) }{ 2 } + ( n + 1 ) \times 1\\
			~&= \frac{ n ( n + 1 ) }{ 2 } + ( n + 1 ) \times \frac{2}{2}\\
			~&= \frac{ n ( n + 1 ) }{ 2 } + \frac{ ( n + 1 ) \times 2 }{2}\\
			~&= \frac{ n ( n + 1 ) }{ 2 } + \frac{ 2 ( n + 1 ) }{2}\\
			~&= \frac{ n ( n + 1 ) + 2 ( n + 1 ) }{2}\\
			~&= \frac{ ( n + 2 ) ( n + 1 ) }{2}\\
			~&= \frac{ ( n + 1 ) ( n + 2 ) }{2}\\
			~&= \frac{ ( n + 1 ) ( ( n + 1 ) + 1 ) }{2}\qed
		\end{align*}
\end{description}

\textbf{Bonus}
Es gilt:
\[ \sum_{ i = 1 }^{ n } i = 1 + 2 + 3 + \dotsb + n - 2 + n - 1 + n. \]
Daraus folgt, dass 
\[ 2 \sum_{ i = 1 }^{ n } i = \]
\begin{alignat*}{7}
	&&& \omit\hfill 1 \hfill &\,+\,& \omit\hfill 2 \hfill &\,+\,& \omit\hfill 3 \hfill &\,+ \dotsb +\,& \omit\hfill n - 2 \hfill &\,+\,& \omit\hfill n - 1 \hfill &\,+\,& \omit\hfill n \hfill\\
	&&+\,& \omit\hfill n \hfill &\,+\,& \omit\hfill n - 1 \hfill &\,+\,& \omit\hfill n - 2 \hfill &\,+ \dotsb +\,& \omit\hfill 3 \hfill &\,+\,& \omit\hfill 2 \hfill &\,+\,& \omit\hfill 1 \hfill\\
	&&=\,& \omit\hfill ( n + 1 ) \hfill &\,+\,& \omit\hfill ( n + 1 ) \hfill &\,+\,& \omit\hfill ( n + 1 ) \hfill &\,+ \dotsb +\,& \omit\hfill ( n + 1 ) \hfill &\,+\,& \omit\hfill ( n + 1 ) \hfill &\,+\,& \omit\hfill ( n + 1 ) \hfill.\\
\end{alignat*}
Da diese Summe aus $ n $ Summanden besteht, die alle $ n + 1 $ sind, folgt
\begin{align*}
	2 \sum_{ i = 1 }^{ n } i & = n ( n + 1 )\\
	\sum_{ i = 1 }^{ n } i & = \frac{n ( n + 1 )}{2}\qed
\end{align*}

\section*{Aufgabe 3}
Wir zeigen mithilfe vollständiger Induktion, dass
\[ \sum_{ k = 1 }^{ n } k^3 = \frac{ n^2 ( n + 1 )^2 }{ 4 } \]
gilt.
\begin{description}
	\item[I.A. ($ n = 1 $)]
		\[ \sum_{ k = 1 }^{ 1 } k^3 = 1^3 = \frac{2^2}{4} = 1^3 \frac{ 1^2 ( 1 + 1 )^2 }{ 4 } \]
	\item[I.S.] ~
		\begin{description}
			\item[I.V.]
				\[ \sum_{ k = 1 }^{ n } k^3 = \frac{ n^2 ( n + 1 )^2 }{ 4 } \]
		\end{description}
		Z.z.:
		\[ \sum_{ k = 1 }^{ n + 1 } k^3 = \frac{ ( n + 1 )^2 ( ( n + 1 ) + 1 )^2 }{ 4 } \]
		Für die linke Seite folgt aus I.V. und Rechengesetzen für Potenzen, Multiplikation und Addition:
		\begin{align*}
			\sum_{ k = 1 }^{ n + 1 } k^3 &= \sum_{ k = 1 }^{ n } k^3 + ( n + 1 )^3\\
			~&= \frac{ n^2 ( n + 1 )^2 }{ 4 } + ( n + 1 )^3 \times \frac{4}{4}\\
			~&= \frac{ n^2 ( n + 1 )^2 }{ 4 } + \frac{4( n + 1 )^3 }{4}\\
			~&= \frac{ n^2 ( n + 1 )^2 + 4( n + 1 )^3 }{4}\\
			~&= \frac{ n^2 {\color{red}( n + 1 )^2} + 4( n + 1 ) {\color{red} ( n + 1 )^2 } }{4}\\
			~&= \frac{ [ n^2 + 4( n + 1 ) ] {\color{red} ( n + 1 )^2 } }{4}\\
			~&= \frac{ [ n^2 + 4n + 4 ] ( n + 1 )^2 }{4}\\
		\end{align*}
		Mithilfe der ersten binomischen Formel lässt sich $ n^2 + 4n + 4 $ auch als $ ( n + 2 )^2 $ schreiben:
		\[ \sum_{ k = 1 }^{ n + 1 } k^3 = \frac{ ( n + 2 )^2 ( n + 1 )^2 }{4} \]
		Aus den Gesetzen für Multiplikation und Addition folgt schließlich:
		\begin{align*}
			\sum_{ k = 1 }^{ n + 1 } k^3 &= \frac{ ( n + 1 )^2 ( n + 2 )^2 }{4}\\
			~&= \frac{ ( n + 1 )^2 ( ( n + 1 ) + 1 )^2 }{4} \qed
		\end{align*}

\end{description}

\end{document}
