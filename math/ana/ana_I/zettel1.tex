\documentclass{gadsescript}

\usepackage[ngerman]{babel}

\settitle{Zettel 1}
\begin{document}
\maketitle
\section{Aufgabe 1}
$ n \coloneqq \text{ Anzahl der Studierenden }, n \in \N $\\
Beh.: Alle Studierende der Analysis 1-Vorlesung haben denselben Namen.\\
\begin{description}
	\item[I.A. ( $ n = 1 $)] ein*e Student*in hat einen Namen
	\item[I.S.]
		\begin{description}
			\item[I.V.] $ n $ Studenten haben denselben Namen
		\end{description}
		
\end{description}

\hrule

Der Fehler liegt darin, dass für $ n = 1 $ die Mengen $ M_1 \coloneqq \{ a_1, \dotsc, a_n \} $ und $ M_2 \coloneqq \{ a_2, \dotsc, a_n+1 \} $, folgendes gilt: $M_1 \cap M_2 = \emptyset$, also gibt es keine Personen die sowohl in $ M_1 $ und $ M_2 $ sind, also müssen die Person(en) in $ M_1 $ und $ M_2 $ nicht denselben Namen tragen.

\end{document}
