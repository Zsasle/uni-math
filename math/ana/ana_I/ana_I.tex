\documentclass{gadsescript}

\usepackage[ngerman]{babel}

\settitle{Analysis I}

\begin{document}
\maketitle

\section{Organisation, Tipps \& Tricks und Literaturhinweise}

Mathe...
\begin{itemize}
	\item ist intellektuell extrem herausfordernd
	\item kommt mit einem hohen Arbeitsaufwand
	\item oft falschen Erwartungen und
	\item ist wie Ausdauersport
\end{itemize}

aber dafür ist Mathe eines der  schönsten Studien c:

Generelles Zeitmanagement:
\begin{itemize}
	\item Vor- und Nachbereitung wahrscheinlich mehr als die gesetzten $14 \times \qty{3}{\hour} = \qty{42}{\hour}$
	\item Klausurvorbereitung auch mehr als $\qty{39}{\hour}$
	\item Pro Woche $ 2 \times \qty{1.5}{\hour}$, $2 \times \qty{2}{\hour} $, $ \qty{1.5}{\hour} $, $ \qty{10}{\hour} $
	\item Es gibt immer eine Aufgabe die man nicht lösen kann
	\item In die Vorlesungen kommen
\end{itemize}

Vorlesung:
\begin{itemize}
	\item normal nicht alles zu verstehen
	\item Notizen was man nicht versteht
	\item Punkte konzise angehen
	\item \textbf{Mathe muss sich gedanklich setzen} - genügend Zeit zu verarbeiten
\end{itemize}

Übungen:
\begin{itemize}
	\item zeitintensiv
	\item Ergebnisse vernünftig aufschreiben
	\item Weg zu einer korrekter Lösung ist sehr langwierig
	\item \textbf{nicht 10 Blätter Papier ab, von denen 9.5 inkonklusiv sind}
	\item also schön Aufschreiben
\end{itemize}

Wenn wir einen Satz gezeigt bekommen, dann bekommen wir nicht die gescheiterten Jahrelangen Versuche zur Schau, sondern nur die Ausgearbeitete Lösung $\rightarrow$ also bei uns auch langer weg, aber Aufschreiben nur klein\\

Übungszettel:
\begin{itemize}
	\item $ 50\% $ muss richtig sein
	\item bis Freitag 10:00 Uhr
	\item in F4
	\item diese Woche nicht so umfangreich, weil weniger Zeit
	\item auf ILIAS Terminfindung Abstimmung
	\item Donnerstag Einteilung in Tutorien
	\item Blätter tackern :c
	\item alle zwei Wochen Beweismechanik Aufgaben, nur digital nicht in Papier (ist dann die letzte Aufgabe)
\end{itemize}

Literaturempfehlung:
\begin{itemize}
	\item Otto Forster: Analysis 1
		\begin{itemize}
			\item kurz und knapp - aber konzise, udn das hilft
			\item ähnliche Struktur wie Vorlesung
			\item weig motivation und wenige Querverbindungen
		\end{itemize}
	\item Königsberger: Analysis 1
		\begin{itemize}
			\item kurz - aber konzise
			\item alle themen der Vorlesung, andere Struktur
			\item mehr motivation und Querverbindungen
		\end{itemize}
	\item Klaus Fritsche: Grundkurs Analysis 1
		\begin{itemize}
			\item ausführlich
		\end{itemize}
	\item Daniel Grieser: Analysis I
		\begin{itemize}
			\item Ausfühlich, aber mit Fokus auf das Wesentliche
			\item alle Themen der Volesung enthalten, ähnliche Struktur
			\item bunt??
		\end{itemize}
	\item Harro Huser: Lehrbuch der Analysis Teil 1
		\begin{itemize}
			\item extrem ausfühlich,dick, an einigen stellen sehr extensiv
			\item alle und mehr Themen als Vorlesung
			\item Querverbindungen
		\end{itemize}
	\item Walter Rudin: Analysis
		\begin{itemize}
			\item sehr knapp und elegant
			\item klassiker
			\item alle themen der Volesung, leicht andere Struktur
			\item empfehlenswertes Buch fortgeschrittene Leser*innen
			\item nicht für Anfänger*innen
		\end{itemize}
	\item Herber amann, Joachim Escher: Analysis I
		\begin{itemize}
			\item strkt logischer Aufbau, damit teils länglich. Großes Bild
			\item alle Themen, andere Struktur
			\item auch nicht für anfänger*innen
		\end{itemize}
	\item Terence Tao: Analysis (englisch, aber gut)
	\item Rober Denk, Reinhard Racke: Kompendium der ANalysis
		\begin{itemize}
			\item kurz und knapp, teils wie Nachschlagewerk
			\item alle themen
		\end{itemize}
	\item Florian Modler, Martin Kreh: Tutorium Analysis 1 und Lineare Algebra 1
		\begin{itemize}
			\item kurz und knapp, teils wie nachschalgewerk
			\item von studierende für studierende
			\item aber enthält ein paar Fehler
		\end{itemize}
\end{itemize}

\newpage
\section{Natürliche Zahlen und elemntare Begriffe}
\subsection{Zahlbereiche}
\[ \N \coloneqq \{ 1, 2, 3, \dotsc \} \]
\[ \N_0 \coloneqq \{ 0, 1, 2, 3, \dotsc \} \]
\[ \Z \coloneqq \{ \dotsc, -3, -2, -1, 0, 1, 2, 3, \dotsc \} \]
\[ \Q \coloneqq \{ \frac{p}{q} : p \in \Z, q \in \N \} \]
\[ \R \coloneqq \{ \text{ reelle Zahlen } \} \]

Wir besprechen gar nicht was eine Menge ist, das ist zu philosophisch\\
Es ist schwierig Mengen zu Definieren, man kommt schnell auf logische Wiedersprüche\\

\begin{itemize}
	\item Notation: für $ x $ schreiben wir für eine Eigenschaft $ A $ ``$ A(x)$'', falls $ x $ $ A $ erfüllt.
	\item[$\rightarrow$] Menge aller Objekte $ x $ mit $ A(x) $
		\[ \{ x : A(x) \} \]
	\item[$\rightarrow$] gibt es kein $ x$ mit $ A(x) $, so nennen wir die Menge leer, ``$\emptyset$''
\end{itemize}

\begin{itemize}
	\item $\exists \hat= \text{ Existenzquantor, ``es existiert''}$
	\item $ A, B, $ Eig., $M\coloneqq \{ x : x \text{ erf. } A \}$\\
		$N \coloneqq \{ x : \text{ erf. } B \} $\\
	$ M \subset N$, falls $ \forall x \in M : x \in N $
	\item $ M = N $, falls $ M \subset N \vee N \subset M $
	\item ``Echte Tielmenge'': $ M \nsubseteqq N$, falls $ M \subset N, N \neq N.$
\end{itemize}

\begin{subexample}[(gerade Zahlen)]
	$ n \in \N_0 \text{ gerade } :\iff ( \exists k \in \N_0 : n = 2k) $\\
	\begin{align}
		M \coloneqq &\{ n \in \N_0 : \exists k \in \N_0 : n = 2k \}\\
		=&\{2k : k \in \N_0
	\end{align}
\end{subexample}

\begin{example}[$ \N \subsetneqq \N_0 \subsetneqq \Z \subsetneqq \Q \subsetneqq \R $]
	Zu $ \Q \subsetneqq \R : \sqrt{2} \notin \Q $. Widerspruchsbeweis: Ang.,
	$ \sqrt{2} \in \Q $, so $ \sqrt{2} = \frac{p}{q} $, mit $ p \in \N_0, q \in \N $. \OE $ p, q $ teilerfremd (d.h. Bruch ist vollständig gekürzt).. Also $ p^2 = 2q^2 $\\
	$\implies$ $ p $ ist gerade. Also $ p = 2l $ mit $l \in N_0$.\\
	$\implies$ $ 4l^2 = p^2 = 2q^2 \implies 2l^2 = q^2 \implies q $ gerade.\\
	$ \implies $ $ p, q $ gerade. $ \implies p, q $ nicht teilerfremd. \qed
\end{example}

\subsection{Vollständige Induktion}
\begin{itemize}
	\item Ziel: Beweis von Aussagen für alle $ n \in \N_0 $
\end{itemize}
\textbf{Dominoprinzip}: Wenn alle Steine umfallen sollen,
\begin{itemize}
	\item müssen wir den 1. Stein umwerfen,
	\item muss stehts der $n$-te Stein den $(n+1)$-ten umwerfen.
\end{itemize}

\textbf{Prinzip (vollst. Ind.)} Wollen wir eine Aussage $ A(n) \forall n \in \N $ zeigen; so zeigen wir
\begin{enumerate}[label=(\roman*)]
	\item $ A(1) $ gilt (Induktionsanfang)
	\item Aus $ A(n) $ für $ n \in \N $ stets $ A(n+1) $ folgt. (Induktionsschritt)
\end{enumerate}

\begin{definition}[Summen]
	Für $x_-1, \dotsc, x_n \in \R $ definieren wir
	\[ \sum_{k = 1}^{n} x_k \coloneqq x_1 + \dotsc + x_n \]
\end{definition}

\begin{example}[Geometrische Summe]
	\[ \forall n \in \N : \]
	\begin{equation}
		\underbrace{\sum_{k=0}^{n} x^k}{x^0 + x^1 + ... + x^n} =  \frac{1-x^{n+1}}{1-x}
	\end{equation}
	
	\begin{description}
		\item[I.A. $n = 1$]
			\[ \sum_{k=0}{1}x^k = x^0 + x^1 = 1+x = \frac{(1-x)(1+x)}{1-x} = \frac{1-x^2}{1-x} \]
		\item[I.S.]
			\[ n \to n+1 \] Angenommen, (equation) gilt für ein $n \in \N $.
			z.z. (equation) gilt für $ n + 1 $
			\[ \sum_{k=0}^{n+1}x^k = \left( \sum_{k=0}{n}x^k\right) + x^{n+1} =  \]
	\end{description}
\end{example}

\end{document}
